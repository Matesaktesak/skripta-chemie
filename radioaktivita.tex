\section{Radioaktivita}
\underline{Uranové paprsky} - objev \underline{Becquerel} (1896)
$\rightarrow$ ozáření fotografické desky (kámen \textbf{smolinec} z Jáchymova)

\vspace{1em}

\underline{Marie Curie Sklodowská} + manžel \underline{Pierre Curie} - objev $_{84}Po$ (polonia) a $_{88}Ra (radia)$

$\rightarrow$ paprsek $=$ \underline{radioaktivita} - V roce 1903 udělení Nobelovy ceny pro Marii, Piera a Becquerela

K maturitě je třeba znát stručný životopis rodiny Curie a Sklodowských.

\subsection{Termíny}

\begin{itemize}
    \item \underline{IZOTOPY}: Stejné Z(protonové \#), liší se počtem neutronů \begin{itemize}
        \item př. $^1_1H$ (vodík, protium), $^2_1H$ (deuterium), $^3_1H$ (tritium)
        \item př. $^{12}_6C$, $^{13}_6C$, $^{14}_6C$ (radioaktivní) $\Rightarrow$ radiouklíkové datování (stanovení stáří organických materiálů)
        \item př. $^{235}_{92}U$, $^{237}_{92}U$, $^{238}_{92}U$ atd.
        \end{itemize}
    \item \underline{IZOBARY}: Jiné Z, stejná A(nukleonové \#) - př. $^{40}_{20}Ca$ a $^{40}_{19}K$
    \item \underline{IZOTONY}: Stejný počet neutronů - př. $^{12}_5B$ a $^{13}_6C$ (oba mají 7$^1_0n$)
\end{itemize}

\subsection{Druhy záření}

$^4_2\alpha \; = \; ^4_2He$ - alfa záření se šíří cca $\frac{1}{10}c$ (rychlosti světla), zachytí se i listem papíru

\vspace{2em}

$\beta$: \begin{itemize}
    \item $\beta^{-} \; = \; ^0_{-1}e$ (elektron) - šíří se cca $\frac{9}{10}c$, záchyt kovovými fóliemi (alobal)
    \item $\beta^{+} \; = \; ^0_{+1}e$ (pozitron)
\end{itemize}

\vspace{2em}

$\gamma$ (gama) $=$ elektromagnetické záření - proud fotonů, rychlost světla, záchyt olověnými deskami, betonem, \underline{zhoubné}

\subsection{Poločas rozpadu T}