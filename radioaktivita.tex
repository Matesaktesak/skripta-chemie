\section{Radioaktivita}
\underline{Uranové paprsky} - objev \underline{Becquerel} (1896)
$\rightarrow$ ozáření fotografické desky (kámen \textbf{smolinec} z Jáchymova)

\vspace{1em}

\textbf{Marie Curie Sklodowská} + manžel \textbf{Pierre Curie} - objev $_{84}Po$ (polonia) a $_{88}Ra$ (radia)

$\rightarrow$ paprsek $=$ \underline{radioaktivita} - V roce 1903 udělení Nobelovy ceny pro Marii, Piera a Becquerela

K maturitě je třeba znát stručný životopis rodiny Curie a Sklodowských.

\subsection{Termíny}

\begin{itemize}
    \item \underline{IZOTOPY}: Stejné Z(protonové \#), liší se počtem neutronů \begin{itemize}
        \item př. $^1_1H$ (vodík, protium), $^2_1H$ (deuterium), $^3_1H$ (tritium)
        \item př. $^{12}_6C$, $^{13}_6C$, $^{14}_6C$ (radioaktivní) $\Rightarrow$ radiouklíkové datování (stanovení stáří organických materiálů)
        \item př. $^{235}_{92}U$, $^{237}_{92}U$, $^{238}_{92}U$ atd.
        \end{itemize}
    \item \underline{IZOBARY}: Jiné Z, stejná A(nukleonové \#) - př. $^{40}_{20}Ca$ a $^{40}_{19}K$
    \item \underline{IZOTONY}: Stejný počet neutronů - př. $^{12}_5B$ a $^{13}_6C$ (oba mají 7$^1_0n$)
\end{itemize}

\subsection{Druhy záření}
    \begin{itemize}
        \item $^4_2\alpha \; = \; ^4_2He$ - alfa záření se šíří cca $\frac{1}{10}c$ (rychlosti světla), zachytí se i listem papíru
        \item $\beta^{-} \; = \; ^0_{-1}e$ (elektron) - šíří se cca $\frac{9}{10}c$, záchyt kovovými fóliemi (alobal)
        \item $\beta^{+} \; = \; ^0_{+1}e$ (pozitron)
        \item $\gamma$ (gama) $=$ elektromagnetické záření - proud fotonů, rychlost světla, záchyt olověnými deskami, betonem, \underline{zhoubné}
    \end{itemize}

\subsection{Poločas rozpadu $T_\frac{1}{2}$}
Lepší název je \underline{poločas přeměny}, jelikož né každá přeměna jádra musí být rozpadem (může se jednat třeba o emisi $\gamma$ záření)

$T_{\frac{1}{2}} \; = \; \frac{ln2}{\lambda}$, konstanta určující dobu, za kterou se rozpadne $\frac{1}{2}$ jader daného prvku $\Rightarrow$ exponenciální graf.
\textbf{$T_{\frac{1}{2}}$} jednodlivých prvků zle najít v tabulkách: 
\begin{itemize}
    \item př. $^{14}_6C \rightarrow T \doteq 5.7$tisíce let
    \item př. $^{208}_{84}Po \rightarrow T \doteq 2.9$roku
    \item př. $^{209}_{84}Po \rightarrow T \doteq 103$let
    \item př. $^{210}_{84}Po \rightarrow T \doteq 138.4$dní
\end{itemize}

\newpage
\subsubsection{Úloha s poločasem rozpadu}

Víme, že při svém vzniku vzorek obsahoval 1 atom $^{14}_6C$ na $10^{12}$ atomů uhlíku $^{12}_6C$ (jelikož tento poměr je v organickém materiálu v atmosféře dlouhodobě stálý)

Při posledním měření bylo ve vzorku nameřen poměř $1 : 1.414*10^{12}$ = $^{14}C \; : \;^{12}C$.

Poločas rozpadu uhlíku $^{14}C$ je 5730let. Jak starý je vzorek?

\rule{5em}{1px}

\begin{itemize}
    \item Původní koncentrace $^{14}C$ ... $c_p = (10^{12})^{-1} = 10^{-12}$
    \item Naměřená koncentrace $^{14}C$ ... $c_m = (1.414 \times 10^{12})^{-1} \doteq 7.07 \times 10^{-13}$
    \item Poločas rozpadu $T_{\frac{1}{2}} = 5730$let
    \item Uplynulá doba od smrti vzorku ... $t = ?$
\end{itemize}

\begin{align*}
    c_{m} &= c_{p} \times \left(\frac{1}{2}\right)^{t \; \div \;  T_{\frac{1}{2}}} \\
    7.07 \times 10^{-13} &= 10^{-12} \times \left(\frac{1}{2}\right)^{t \; \div \;  5730} \\
    \frac{7.07 \times 10^{-13}}{10^{-12}} &= \left(\frac{1}{2}\right)^{t \; \div \;  5730} \\
    \log_{\frac{1}{2}}\left(\frac{7.07 \times 10^{-13}}{10^{-12}}\right) &= t \; \div \;  5730 \\
    t &= \log_{\frac{1}{2}}\left(\frac{7.07 \times 10^{-13}}{10^{-12}}\right) \times 5730 \\
    t &\doteq 2866 let
\end{align*}

Vzorek tedy přestal příjmat atmosferický uhlík před $\doteq$2866lety.

\TabPositions{0em, 15em, 30em}
\subsection{Rozpadové řady}
Přirozené:
\begin{enumerate}
    \item URANOVÁ: \tab $^{238}_{92}U \; \cdots \longrightarrow \; ^{206}_{82}Pb$ \tab A = 4n + 2
    \item THORIOVÁ: \tab $^{232}_{90}Th \cdots \longrightarrow \; ^{208}_{82}Pb$ \tab A = 4n
    \item AKTINOURANOVÁ: \tab $^{235}_{92}U \; \cdots \longrightarrow \; ^{207}_{82}Pb$ \tab A = 4n + 3 
\end{enumerate}

Umělá:
\begin{enumerate}
    \setcounter{enumi}{3}
    \item NEPTUNIOVÁ: \tab $^{237}_{93}Np \cdots \longrightarrow \; ^{205}_{81}Tl$ \tab A = 4n + 1
\end{enumerate}

\subsubsection*{Příkad}
Do které řady patří $\; ^{234}_{92}U$ ?

\begin{align*}
    234&\div 4 = 58 \\
    34& \\
    \underline{2}& \; \longleftarrow \mbox{4n + \underline{2}} \; \Rightarrow \mbox{Uranová řada}\\
\end{align*}

Uran234 patří do uranové řady, protože zbytek po dělení jeho A (nukleonového \#) čtyřmi je 2.

\subsection{Umělá radioaktivita}
dcera \textbf{Irene Curie} + manžel \textbf{F.J.Curie}
Vznik umělých radioizotopů (medicína, konzervace potravin, sterilizace materiálů...)

\[^{27}_{13}Al + \; ^4_2\alpha \; \longrightarrow \; ^{30}_{15}P + \; ^1_0n\]
\vspace{1em}
\[^{238}_{92} +  \; ^1_0n \; \longrightarrow \; ^{237}_{92}U + \; 2^1_0n\]
\vspace{0.5em}

\textbf{Součet čísel na obou stranách se MUSÍ rovnat}

\rule{44em}{1px}

\begin{multicols}{3}
    \vspace{0.5em}
    proton: $^1_1p$
    
    \vspace{0.5em}
    pozitron: $^0_1e$
    
    \vspace{0.5em}
    $^4_2\alpha = ^4_2He$
    
    \vspace{0.5em}
    neutron: $^1_0n$
    
    deuterium: $^2_1D = ^2_1H$
    
    $\beta^{-} = ^0_{-1}e$
    
    elektron: $^0_{-1}e$
    
    tritium: $^3_1T = ^3_1H$
    
    $\beta^{+} = ^0_1e = pozitron$
\end{multicols}

\subsection{Posuvové zákony}
Vytváří-li prvek:
\begin{itemize}
    \item $^4_2\alpha \; \Longrightarrow \;$ A - 4, Z - 2
    \item $\beta^- \; \Longrightarrow \;$ A, \;Z + 1
    \item $\beta^+ \; \Longrightarrow \;$ A, \;Z - 1
\end{itemize}
\textbf{Příklad:} Napiš produkty přeměn:
\begin{enumerate}
    \item rozpadem $\alpha$: \tab \(^{226}_{88}Ra \; \rightarrow \; ^4_2\alpha \; + \; ^{222}_{86}Rn\)
    \item rozpadem $\beta^-$: \tab \(^{32}_{15}P \; \rightarrow \; ^0_{-1}e + ^{32}_{16}X\)
    \item rozpadem $\beta^+$: \tab \(^{11}_{6}C \; \rightarrow \; ^0_{1}e + ^{11}_{5}X\)
\end{enumerate}

\subsection{Jaderné reakce}
Musí být dodržen:
\begin{itemize}
    \item \textbf{zákon zachování energie}
    \item zákon zachování hybnosti
    \item zachování elektrického náboje
    \item zachování počtu nukleonů
\end{itemize}

Dělení: transmutace, štepení, fůze

\subsubsection{Transmutace}
Reakce při nichž se mění jádro prvku na jiné, které se liší maximálně o \textbf{2 v Z} a o \textbf{4 v A}
Příklady:
\[^{209}_{83}Bi \; + \; ^4_2\alpha \; \longrightarrow \; ^{211}_{85}Az \; + \; 2^1_0n\]
\vspace{0.2em}
\[^{41}_{19}K \; + \; ^2_1D \; \longrightarrow \; ^{42}_{19}K \; + \; ^1_1p\]
\vspace{0.2em}
\[^{10}_{5}B \; + \; ^1_0n \; \longrightarrow \; ^{7}_{3}Li \; + \; ^4_2\alpha\]

\subsubsection{Stěpení jader}
Reakce při nichž se štěpí těžká jádra na (obvykle) 2 středně těžká jádra + neutron(y) + velké množství energie (v MeV - megaelektronvolt)
Příklady:
\[^{235}_{92}U \; + \; ^1_0n \; \longrightarrow \; _{56}Ba \; + \; _{36}Kr \; + \; 3^1_0n\]
\vspace{0.1em}
\[^{235}_{92}U \; + \; ^1_0n \; \longrightarrow \; _{54}Xe \; + \; _{38}Sr \; + \; 2^1_0n\]
Jádra se štěpí s určitou pravděpodobností

\subsubsection{Řetězová reakce}
Potvrzeno na \underline{jaře 1939}

Ze štěpení jádra atomem se uvolňují další neutrony, které štěpí další atomy atd.

Jako palivo se běžně používá izotop $^{235}_{92}U$, občas také $^{239}_{94}Pu$ (plutonium)

Řetězová štěpná reakce je kromě atomových elektráren také podstatou jaderné bomby.

\vspace{1em}

\underline{ENRICO FERMI} 2.12.1942 poprvé uskutečnil \underline{řízenou řetězovou reakci} (v jaderném reaktoru na hřišti univezity v Chicagu).
Fermi je nositelem Nobelovy ceny za z roku 1938 za přípravu 1. transuranu pPrvku s vyšší protonovým číslem než uran) \textbf{Z = 93 $\rightarrow$ Np}

\subsubsection{Projekt Manhattan}
%TODO: Dopsat
"Otec" atomové bomby: Robert Oppenheimer

Dále na ní pracovali například: Fermi, Bohr, Einstein, Feinman, Meitner (žena), Heisenberg, Landau, Kurčatov, Gamow 

První užití jaderné zbraně: červenec 1945 Hirošima, poté Nagasaki

\vspace{1em}
\underline{Termíny}:
\begin{itemize}
    \item \underline{obohacování uranu} izotopem $^{235}_{92}U$ (mezinárodní dohody zakazují nad 5\%)
    \item \underline{kritické množství} (critical mass) $^{235}_{92}U$ je zhruba 44.5kg (koule o průměru 16.8cm)
    \item \underline{atomový reaktor}
    \item úložiště jaderného odpadu
    \item \underline{moderátor} v jaderné elektrárně: snižuje rychlost volných neutronů: grafit, parafin, $D_2O$, sloučeny boru
    \item Těžká voda = $D_2O$ - voda obsahující izotop vodíku Deuterium ($^2_1D$) - má jiné fyzikální i chemické vlastnosti. M = 20, jiné body tání a mrznutí... Organizmy v ní nepřežívají
\end{itemize}

\underline{Jaderné elektrárny}:
\begin{itemize}
    \item Jaderná elektrárna Dukovany (ČR, v provozu od 1985)
    \item Jaderná elektrárna Temelín (ČR, v provozu od 2002)
    
    \item Jaderná elektrárna Chornobyl (Černobyl) - na Ukrajině, velká havárie 26.dubna 1986 - výbuch 4. jaderného bloku během experimentů s jeho odstavováním.
    Poblíž (3km) leží město Pripjať
\end{itemize}

\subsubsection{Jaderná fůze}
Též \underline{jaderná syntéza}, termonukleární reakce

Skládání jader na jádra těžší.

Samovolně probíhá ve hvězdách, například ve Slunci (zatím na He).

Uvolňuje se \underline{obrovské} množství energie. Spývají jádra bez elektronového obalu

\vspace{1em}
Příklady:
\[^2_1D \; + \; ^2_1D \; \longrightarrow \; ^3_2He \; + \; ^1_0n\]
\[^7_3Li \; + \; ^1_1H \; \longrightarrow \; 2^4_2He\]
\[^7_3Li \; + \; ^2_1D \; \longrightarrow \; 2^4_2He \; + \; ^1_0n\]
\[^2_1D \; + \; ^3_1T \; \longrightarrow \; ^4_2He \; + \; ^1_0n\]

\vspace{1em}
Reaktory jsou v US a na jihu Francie. Zatím neumíme fůzy řídit.

\vspace{1em}

Výhody: dostatek surovin (D,T), není odpad - jen netečné He, není radioaktivní (jen $^3_1Y$), bezpečnost - zdá se, že se jedná o ideální zdroj energie.