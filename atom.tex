\setlength{\parindent}{0px}
\section{Atom}
\subsection{Erwin Schrödinger}
Rakouský fyzik (1889 - 1961)

Definoval \underline{ORBIT = ORBITAL} jako místo s 96\% pravděpodobností výskytu $e^-$

Matematicky vyjádřil vlnovou funkci $\Psi$ (psí)

Nositel Nobelovy ceny za fyziku 1933

\TabPositions{0em, 12em, 25em}
\subsection{Kvantová čísla}
\begin{description}
    \item[hlavní \textbf{n}] \tab 1-$\infty$(zatím 7) \tab udává \underline{energii} orbitu
    \item[vedlejší \textbf{l}] \tab 0-(n-1) \tab udává \underline{tvar} orbitu
    \item[magnetické \textbf{m}] \tab -l...0...+l \tab udává \underline{počet orbitalů} a jejich orientaci
    \item[spinové \textbf{s}] \tab $-\frac{1}{2}$ - $\frac{1}{2}$ \tab udává \underline{spin} $e^-$
\end{description}

\centering
\subsubsection{Slupky, energetické hladiny (dráhy)}
\begin{multicols}{2}
    $n = 1 \to K$

    $n = 2 \to L$
    
    $\vdots$
    
    $n = 3 \to M$
    
    $n = 4 \to N$

    $\vdots$ 
\end{multicols}

\centering
\subsubsection{Podslupky}
\begin{multicols}{2}
    $l = 0 \to s$

    $l = 1 \to p$
    
    $l = 2 \to d$
    
    $l = 3 \to f$
\end{multicols}


\subsubsection{Tvary orbitů}
\vspace{2em}
\raggedright
\TabPositions{0em, 20em}
$l = 0 \to$ tvar orbitu \textbf{s}: kulově symetrický
\tab
$\overbrace{\underbrace{1s}_{\cdot}}^{\text{\tiny{hlavní kv. \#, vedlejší \#}}}  \underbrace{2s}_{\circ} \hspace{3em} \underbrace{3s}_{\bigcirc}$

\vspace{2em}
\raggedright
$l = 1 \to $ tvar orbitu \textbf{p}: "ležatá osmička" \tab \hspace{2em} \huge{$\infty$}

\normalsize

$l = 2 \to $ tvar orbitu \textbf{d}: "čtyřlístek"

\vspace{2em}

$l = 3 \to $ tvar orbitu \textbf{f}: "velmi složitý tvar"


\raggedright

\subsection{Výstavbový princip}
\subsubsection[Znázornění orbitů a elektronů]{\underline{Znázornění orbitů a elektronů} v nich ($\downarrow \uparrow,~ \uparrow \uparrow,~ \downarrow\downarrow$)}
\begin{description}
    \TabPositions{0em, 12em}
    \vspace{1em}
    \item[a)] prostorovým tvarem: \tab s, p, d, f
    \vspace{2em}
    \item[b)] psaným symbolem: \tab \(
        \begin{array}{c}
            \qquad\qquad\;\; e^-\\
            \qquad \;\;\; \nearrow \\
            \;\;3d^7 \\
            \swarrow~\searrow \\
            \mbox{n=3} \qquad \mbox{l=2}
        \end{array}
    \)
    \hspace{6em}
    \(
        \begin{array}{c}
            \qquad\qquad\;\; e^-\\
            \qquad \;\;\; \nearrow \\
            \;\;4p^5 \\
            \swarrow~\searrow \\
            \mbox{n=4} \qquad \mbox{l=1}
        \end{array}
    \)
    \vspace{2em}
    \item[c)] rámečky: \tab  : 3 \fbox{$\downarrow\uparrow$}\fbox{$\downarrow\uparrow$}\fbox{$\downarrow$ }\fbox{$\downarrow$ }\fbox{$\downarrow$ } \hspace{3em} 4 \fbox{$\downarrow\uparrow$}\fbox{$\downarrow\uparrow$}\fbox{$\downarrow$ }
\end{description}

\newpage
\addcontentsline{toc}{subsubsection}{Příklad}
Příklad: Urči maximální počet $e^-$ ve slupce \textbf{N}

\smallskip
\TabPositions{0em, 7em}
N $\Rightarrow$ n=4 $\Rightarrow$\tab 0(s) $\Rightarrow$ m=0 (1 orbit)

    \tab 1(p) $\Rightarrow$ m=-1,0,1 (3 orbity)

    \tab 2(d) $\Rightarrow$ m=-2,-1,0,1,2 (5 orbity)

    \tab 3(f) $\Rightarrow$ m=-3,-2,-1,0,1,2,3 (7 orbity)

\medskip
Dohromady 16 orbitů * 2$e^-$ \textbf{= 32$e^-$}

...jelikož v každém orbitu mohou být 2 elektrony s opačným spinem (tzv. Pauliho vylučovací princip)

\bigskip
\fbox{\phantom{$\uparrow$ }} \underline{prázdný orbit = vakantní}

\subsubsection{Pravidla zaplňování orbitů}
\begin{enumerate}
    \item Pauliho vylučovací princip
    \item Hundovo pravidlo: Nejprve se zaplňují orbity jedním $e^- \; \Rightarrow$ nespárované $e^-$ mají stejný spin
    
    Příklad: $3d^7$: 3 \fbox{$\downarrow\textcolor{magenta}{\uparrow}$}\fbox{$\downarrow\textcolor{magenta}{\uparrow}$}\fbox{$\downarrow\textcolor{magenta}{.}$}\fbox{$\downarrow\textcolor{magenta}{.}$}\fbox{$\downarrow\textcolor{magenta}{.}$}
    
    Jedná se o tzv. \underline{DEGENEROVANÉ} orbity (mají stejné \underline{n} a \underline{l}, liší se v m) $\Rightarrow$
    
    $\Rightarrow$ orbity \textbf{s} nesjou degenerované, orbity \textbf{p} jsou 3x degenerované, orbity \textbf{d} 5x, \textbf{f} 7x
    \item Výstavbový princip: nejprve se zaplňují orbity s nízkou energií $\doteq$ v tomto pořadí:
    
    1s, 2s, 2p, 3s, 3p, 4s, 3d, 4p, 5s, 4d, 5p, 6s, 5d, 4f, 6p, 7s, 6d \dots
    \item Pravidlo \textbf{n+l}: Když je součet n+l stejný, zaplňují se provně orbity s nižší hodnotou \underline{n}.
\end{enumerate}

\subsubsection{Elektronové konfigurace podle výstavbového principu}
$\underline{\underline{_{13}}}Al$: $1s^2$, $2s^2$, $2p^6$, $3s^2$, $3p^1$ (součet $e^- = \underline{\underline{13}}$)

$\underline{\underline{_{26}}}Fe^-$: $1s^2$, $2s^2$, $2p^6$, $3s^2$, $3p^6$, $4s^2$, $3d^{\underline{7}}$ (součet $e^- = \underline{\underline{27}}$ - protože se jedná o záporný iont, má $e^-$ navíc!)

\subsubsection[Zápis se vzácným plynem]{Elektronové konfigurace podle předcházejícího vzácného(inertního) plynu - 8.hlps}
%\vspace{1em}
\(\underbrace{_{16}S\left[_{10}Ne\right]}_{16 - 10 = 6e^-}: \textcolor{red}{3}s^2, 3p^4 \longrightarrow 
    \mbox{\textcolor{red}{n} = zároveň \underline{\# periody} ve které se prvek nachází (S je ve 3. řádku PSP.)}
\)

\vspace{1em}
Vždy se začíná orbitem \textbf{s} a pak další v pořadí \hyperlink{vystavbovyprincip}{výstavbového principu}

\vspace{2em}
$\underbrace{_{35}Br\left[_{18}Ar\right]}_{35 - 18 = 17e^-}$ : $4s^2, 3d^{10}, 4p^5$

\vspace{2em}
$\underbrace{_{53}I\left[_{36}Kr\right]}_{57 - 36 = 17e^-}$ : $5s^2, 4d^{10}, 5p^5$

\subsubsection{Elektronové konfigurace podle valenčních elektronů}
Valenční vrstva(svéra, též hladina) je poslední od jádra pro daný atom

\begin{description}
    \item[a)] \textbf{Konfigurace základních (hlavních) prvků} (I.A - VIII.A):
    
        Valenční $e^-$ zaplňují \underline{ns a np}. (Kontrola hlavního kvantového \# = \# periody)
        
        \medskip
        \textbf{\underline{Počet valenčních $e^-$ = číslo skupiny} ve které se prvek nachází.} Například:

        \(_{13}Al: 3s^2, 3p^1\): \quad 3 \fbox{$\downarrow\uparrow$}, 3 \fbox{$\downarrow$ }\fbox{\phantom{$\uparrow$ }}\fbox{\phantom{$\uparrow$ }} $\longleftarrow$ celkem 3 $e^- \Rightarrow$ 3.hlavní podskupina
        
        \(_{10}Ne: 2s^2, 2p^6\): \quad 2 \fbox{$\downarrow\uparrow$}, 2 \fbox{$\downarrow\uparrow$}\fbox{$\downarrow\uparrow$}\fbox{$\downarrow\uparrow$} $\longleftarrow$ plné orbity = inertní plyn
        
        \begin{multicols}{2}
            \(_{50}Sn: 5s^2, 5p^2\): \quad 5 \fbox{$\downarrow\uparrow$}, 5 \fbox{$\downarrow$ }\fbox{$\downarrow$ }\fbox{\phantom{$\uparrow$ }}
    
            \(_{12}Mg: 3s^2\): \quad 3 \fbox{$\downarrow\uparrow$}
        \end{multicols}

        \(_{6}C: 2s^2, 2p^2\): \quad 2 \fbox{$\downarrow\uparrow$}, 2 \fbox{\textcolor{blue}{$\downarrow$} }\fbox{\textcolor{blue}{$\downarrow$} }\fbox{\phantom{$\uparrow$ }} $\longrightarrow$
        \(_{6}C^*: 2s^2, 2p^2\): \quad 2 \fbox{\textcolor{red}{$\downarrow$} }, 2 \fbox{\textcolor{red}{$\downarrow$} }\fbox{\textcolor{red}{$\downarrow$} }\fbox{\textcolor{red}{$\downarrow$} }

        Uhlík se vyskytuje jako \textcolor{blue}{2-vazný} jen v CO (C=O), jinak je vždy \textcolor{red}{4-vazný}
    
        * = excitovaný stav $\to \; e^-$ přecházejí na vyšší energetické hladiny do nejbližšího vakantního(prázdného orbitu) v pořadí s$\to$p$\to$d$\to$f
    
    \item[b)] \textbf{Konfigurace přechodných prvků} (skupiny B)

        Valenční elektrony lezí v \underline{$ns^{0-2}, (n-1)d^{1-10} \; \longrightarrow$} tzv. \underline{d} prvky
        
        Jejich konfigurace není zcela pravidelná a často se od systému liší. Například:
        \begin{multicols}{2}
            \(_{29}Cu: 4s^1, 3d^{10}\)
    
            \(_{46}Pd: 5s^0, 4d^{10}\)
    
            \(_{24}Cr: 4s^1, 3d^{5}\)
    
            \(_{23}V: 4s^2, 3d^{3}\)
        \end{multicols}

    \item[c)] \textbf{Konfigurace vnitřně přechodných prvků} (lanthanoidy, aktinoidy)

        Prvky \underline{f}, kde valenční elektrony leží v \underline{$ns^2, (n-1)d^{0-2}, (n-2)f^{0-14}$}

        Tyto vrstvy jsou poznaménány značnýmy nepravidelnostmi v obsazování orbitů...

\end{description}

\subsection{Jádro atomu}
objev \underline{jádra}: \underline{RUTHERFORD} (1911-1920), planetární model apod.

+ objev \underline{protonu} v jádře. Po něm provek $_{104}Rf$(Rutherfordium) v PSP.

\vspace{1em}

objev \underline{neutronu} v jádře: THOMSON (1932)

\[^9_4Be\; + \;^4_2\alpha \; \rightarrow \; ^{12}_6C \; + \;^1_0n\]

+ objevy dalších částic, které se dělí do skupin apod.: \underline{bosony, fermiony, hadrony, kvarky}, piony

\vspace{1em}

Jádro se skládá z protonů a neutronů - počet \textbf{protonů se uvání jako levý spodní index}, zatímco celkový počet částic v jádře(nukleonové číslo, \textbf{protony+neutrony) se uvádí v levém horním indexu}