\section{Chemická vazba}
Atomy se k sobě přibližují na optimální vzdálenost až se překryjí jejich valenční orbity a vznikne vazebný elektronový pár.

\underline{Při vzniku} chemické vazby \underline{se energie uvolňuje} $\rightarrow$ Stabilita

\subsection{Znázornění chemických vazeb}
\TabPositions{0em, 15em, 30em}
\begin{enumerate}
    \item prostorovým tvarem orbitů
    \item valenční čárkou \tab H + H $\longrightarrow$ $H_2$ \tab H---H
    \item rámečky \tab H + H $\longrightarrow$ $H_2$ \tab \fbox{$\rightarrow$}---\fbox{$\leftarrow$}
\end{enumerate}

\subsection{Kovalentní vazba} - společně sdílejí $e^-$

\vspace{1em}
\underline{ELEKTRONEGATIVITA} je schopnost atomu přitahovat vazebné $e^-$.

V periodách roste, ve sloupcích klesá

Jedná se o bezrozměrné číslo (nemá jednotku)

\subsubsection{Nepolárně kovalentní}
Rozdíl elektronegativit mezi vázanými atomy od 0 - 0,4

\subsubsection{Polárně kovalentní}
Rozdíl elektronegativit mezi vázanými atomy od 0,4 - 1,7

Například $H^{\delta+}$---$Cl^{\delta-}$ $\longleftarrow \; \delta$=delta, částečný, parciální náboj

\subsubsection{Iontová vazba}
Rozdíl elektronegativit mezi vázanými atomy \>1,7

Příklad: \(KBr \; \longrightarrow \; K^+ \; + \; Br^-\) nebo \(NaCl \; \longrightarrow \; Na^+ \; + \; Cl^-\)

\subsubsection{Koordinačně kovalentní}
Vazba \underline{DONOR-AKCEPTOR} - vazba mezi donorem (dárcem) a akceptorem (příjemcem) elektronového páru

\vspace{1em}
Například: $NH_4^+$, $H_3O^+$
% \begin{align*}
%     _7N: \qquad 2s^2 &, \qquad 2p^3 \\
%     \text{\downarrow\uparrow}
% \end{align*}

\subsection{Dělení kovalentních vazeb podle počtu}
\subsubsection*{Vazba jednoduchá}
Vazba $\sigma$ (sigma), leží na spojnici středů vázaných atomů: $\odot$---$\odot$

\subsubsection*{Násobné vazby}
\begin{enumerate}[label=\alph*)]
    \item \textbf{v. dvojná}: jedna vazba $\sigma$ a jedna vazba $\pi$ (pí), ležící mimo spojnici středů jader
    \item \textbf{v. trojná}: jedna vazba $\sigma$ a DVĚ vazby $\pi$
    \item Teoreticky existují i více násobné vazby, v běžné chemii se však nevyskytují
\end{enumerate}

\subsection{Štěpení vazeb}
\begin{enumerate}
    \item \underline{Homolitické} $\longrightarrow$ RADIKÁLY (částice s volným $e^-$)
    
    Například: $CH_3$---$CH_3 \; \longrightarrow \; H_3C\cdot \qquad \cdot CH_3$

    \item \underline{Heterolitické} $\longrightarrow$ IONTY (jedna částice přebere celý elektronový pár)
    
    Například: $Na$---$Cl \; \longrightarrow \; Na^+ \; + \; Cl^-$

\end{enumerate}

\subsection{Kovová vazba}
Elektrony jsou delokalizované (nemají svoje přesné, pevné místo - dříve tzv. "elektronový plyn")

Například: $_3Li$ \tab Jeden valenční elektron "drží" 8 partnerů:
\[\chemfig{Li(-[:0]\bullet)(-[:90]\bullet)(-[:180]\bullet)(-[:270]\bullet)(-[:45]\bullet)(-[:135]\bullet)(-[:225]\bullet)(-[:315]\bullet)}\]

Tzn. že jedna vazba Li---Li je tvořena $\frac{1}{4}e^-$

\subsection{Charakteristika vazeb}
Vazba má svojí:
\begin{itemize}
    \item \underline{DÉLKU} (v nm), nejdelší je vazba jednoduchá, nejkratší pak trojná.
    \item \underline{VAZEBNOU ENERIGII} (v kJ/mol), je to stejná energie, která se uvolní při vziku vazby.
    Největší má vazba trojná, nejmenší jednoduchá.
\end{itemize}

\subsection{Slavá vazebné interakce}
$\doteq$ 10x slabší než kovalentní vazba. Stojí na interakci DIPÓL---DIPÓL

\subsubsection{Van der Waalsovy síly}
Například v grafitu, v nukleových kyselinách a bílkovinách.

\subsubsection{Vodíkové můstky}
H---m

Vazba mezi vodíkem a elektronegativním prvkem (například N, O, F...)

Stabilizují molekuly, ovlivňují jejich chemické vlastnosti.

Vyskytují se v $H_2O$, HF, bazích nukleových kyselin, bílkovinách, $NH_3$ atd.

\subsubsection*{Ovlivňování ostatních molekul}
Molekuly vody se navzájem ovlivňují
\[\chemfig{O(>[:-38]H(>:[:-45]O(>[:-38]H(>:[:-45]))(>[:-142]H(>:[:-135]))))(>[:-142]H(>:[:-135]O(>[:-38]H(>:[:-45]O(>[:-38]H(>:[:-45]))(>[:-142]H(>:[:-135]))))(>[:-142]H(>:[:-135]))))}\]