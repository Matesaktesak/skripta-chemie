\section{Chemická vazba}
Atomy se k sobě přibližují na optimální vzdálenost až se překryjí jejich valenční orbity a vznikne vazebný elektronový pár.

\underline{Při vzniku} chemické vazby \underline{se energie uvolňuje} $\rightarrow$ Stabilita

\subsection{Znázornění chemických vazeb}
\TabPositions{0em, 15em, 30em}
\begin{enumerate}
    \item prostorovým tvarem orbitů
    \item valenční čárkou \tab H + H $\longrightarrow$ $H_2$ \tab H---H
    \item rámečky \tab H + H $\longrightarrow$ $H_2$ \tab \fbox{$\rightarrow$}---\fbox{$\leftarrow$}
\end{enumerate}

\subsection{Kovalentní vazba} - společně sdílejí $e^-$

\vspace{1em}
\underline{ELEKTRONEGATIVITA} je schopnost atomu přitahovat vazebné $e^-$.

V periodách roste, ve sloupcích klesá

Jedná se o bezrozměrné číslo (nemá jednotku)

\subsubsection{Nepolárně kovalentní}
Rozdíl elektronegativit mezi vázanými atomy od 0 - 0,4

\subsubsection{Polárně kovalentní}
Rozdíl elektronegativit mezi vázanými atomy od 0,4 - 1,7

Například $H^{\delta+}$---$Cl^{\delta-}$ $\longleftarrow \; \delta$=delta, částečný, parciální náboj

\subsubsection{Iontová vazba}
Rozdíl elektronegativit mezi vázanými atomy \>1,7

Příklad: \(KBr \; \longrightarrow \; K^+ \; + \; Br^-\)