\newenvironment{boldEnum}
  {\begin{enumerate}
   \let\olditem\item
   \long\def\item##1\\{\olditem{\bfseries ##1}\par}}
  {\end{enumerate}
}

\subsection{Maturitní otázky}
Tento seznam maturitních otázek je \textbf{pouze orientační!} - jelikož se může rok od roku měnit (naposledy aktualizován 2023) a především,
jelikož \textbf{každá škola má vlastní seznam}. Tento seznam je (nebo byl) platný na Gymnáziu Joachima Barranda v Berouně.
Maturitní otázky jsou hlavním tématem pro zkoušení, ale nejsou zcela vyčerpávající. Je vhodné naplnit čas zkoušení i tématy, která s otázkou souvisí.
Například v otázce Radioaktivita je rozhodně žádoucí promluvit i o stavbě atomu a elementárních částicích (zde je možná i odbočka do fyziky). 

%TODO: Prolinkovat omaturitní otáky na témata
\begin{boldEnum}
    \item Stavba atomu \\    
    základní chem. zákony, charakteristika element. částic vývoj představ o stavbě atomu, kvantová čísla,
    orbit elektronová konfigurace, pravidla zaplňování orbitů
    
    \item Radioaktivita \\
    přirozená a umělá radioaktivita druhy záření, posuvové zákony radioaktivní řady, atomový reaktor

    \item Chemická vazba \\    
    vznik chem. vazby druhy chem. vazeb a jejich charakteristika štěpení vazby

    \item Chemické reakce \\    
    energetický průběh reakce, rychlost chem. reakce typy chem. reakcí /proteolytické, redoxní, srážecí,
    komplexotvorné / termochemie, termochemické zákony

    \item Acidobazický děj \\
    teorie kyselin a zásad Brőnstedova teorie kyselin a zásad amfoterní látky, disociační konstanta kyselin
    a zásad vicesytné kyseliny, autoprotolýza, iontový součin vody Sőrensenův vodíkový exponent,
    hydrolýza

    \item Redoxní děj \\    
    oxidace, redukce, oxidační, redukční činidla, redoxní pár výpočet, koeficientů redoxních rovnic,
    disproporcionalizační reakce Beketovova řada kovů, elektrolýza

    \item Mendělejevův periodický systém \\    
    periodický zákon, jeho aplikace /velikost atomu, elektronegativita, ionizační energie/ kovy, nekovy
    chem. značky, vzorce, chemické názvosloví stechiometrické výpočty
    
    \item Kyslík, vodík a jejich sloučeniny \\    
    výskyt, reakce, význam voda, peroxid vodíku vliv na životní prostředí
    
    \item Prvky s \\    
    charakteristika prvků 1. a 2. hl. podskupiny jejich reakce, sloučeniny, význam
    
    \item Prvky p1- p3 \\    
    charakteristika prvků 3., 4., 5. hl. podskupiny reakce, sloučeniny, význam
    
    \item Prvky p4- p8 \\    
    charakteristika prvků 6., 7., 8. hl. podskupiny jejich reakce, sloučeniny, význam
    
    \item Prvky d, f \\    
    charakteristika přechodných a vnitřně přechodných prvků reakce, sloučeniny, význam komplexní
    sloučeniny
    
    \item Základní pojmy organické chemie \\
    struktura, konformace, izomerie, optická aktivita, činidla, typy reakcí v organické chemii, reakční
    mechanismy
    
    \item Nasycené a nenasycené uhlovodíky \\    
    charakteristika, vlastnosti, reakce, význam, alkanů, alkenů, alkadienů, alkinů názvosloví
    
    \item Aromatické uhlovodíky \\    
    struktura benzenu, substituenty 1. a 2. třídy reakce, význam aromátů, názvosloví
    
    \item Halogenderiváty a dusíkaté deriváty uhlovodíků \\
    charakteristika, reakce, význam, názvosloví halogenderivátů, nitrosloučenin, aminů, azosloučenin
    
    \item Deriváty uhlovodíků obsahujících kyslík / síru \\    
    charakteristika, reakce, význam, názvosloví alkoholů, fenolů, etherů a jejich sirných obdob
    
    \item Karbonylové sloučeniny \\    
    charakteristika, reakce, význam, názvosloví aldehydů, ketonů, chinonů.
    
    \item Karboxylové kyseliny \\    
    charakteristika, reakce, význam, názvosloví, soli karboxyl. kyselin, acyl k.k. substituční a funkční
    deriváty k.k.
    
    \item Chemie přírodních látek \\    
    lipidy, terpeny, steroidy, heterocyklické sloučeniny alkaloidy, drogová závislost
    
    \item Základní stavební látky organismů \\    
    biogenní prvky, chemické znaky živých soustav lipidy, sacharidy, aminokyseliny, bílkoviny
    
    \item Metabolické přeměny zákl. organ. látek \\    
    metabolismus, ATP glykolýza, Krebsův cyklus, beta oxidace mastných kyselin, ornithinový cyklus
    
    \item Sacharidy \\    
    fotosyntéza, dělení, vlastnosti, analytické důkazy, cyklické formy sacharidů, významné
    monosacharidy, disacharidy, polysacharidy glykolýza
    
    \item Základní děje v organismech \\    
    fotosyntéza, chemosyntéza proteosyntéza nukleové kyseliny
    
    \item Regulace biochemických dějů \\    
    hormony, enzymy, vitamíny
\end{enumerate}