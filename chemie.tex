\documentclass{article}

\usepackage[a4paper,margin=2.5cm]{geometry}
\usepackage[utf8]{inputenc}
\usepackage[czech]{babel}

\usepackage{graphicx, amsmath, pdflscape, ragged2e, float, array, tabto, longtable, multicol, xcolor, chemfig}
\usepackage[LGRgreek]{mathastext}
\usepackage[hidelinks]{hyperref}

\usepackage{enumitem}

\def\WATER{H_2O}
%\def\HYDRO{#1}{\left(OH\right)_{``#1''}}
    
% ----------------------- DOCUMENT START -----------------------------


\title{Chemie k maturitě}
\author{Stanislava Pojerová\thanks{Sazba: Matyáš Levíček}}
\date{2020-2023}

\begin{document}
    \maketitle

    \begin{abstract}
        Pouhý přepis zpracovaného materiálu paní učitelky RNDRr. Stanislavy Pojerové. Původní materiál je souborem pro kvintu a sextu víceletého gymnázia a byl zpracován během pandemie Covidu 19 v letech 2020 a 2021.
        
        Skripta v této podobě mají sloužit především studentům plánujícím maturitu z chemie. 
    \end{abstract}

    \newpage
    \renewcommand{\contentsname}{Obsah podle tématu}
    \tableofcontents
    \newpage
    
    \section{Úvod}
    Skripta pokrývají učivo nutné pro obstání u profilové zkoušky z chemie. Odvíjejí se od otázek k tomuto předmětu z kánonu Gymnázia Joachyma Barranda v Berouně.

    Učivo je systematizováno v pořadí, které odpovídá výkladu na semináři Systematizace poznatků z chemie v oktávě na GJB. 

    Výše je však kromě obsahu také obsah seřazený podle maturitních otázek - doporučuji proto elekronickou podobu, která umožňuje mezi tématy skákat přes hyperlinky a výrazně tak zjednodušuje orientaci v materiálu.

    \setlength{\parindent}{0px}
\section{Atom}
\subsection{Erwin Schrödinger}
Rakouský fyzik (1889 - 1961)

Definoval \underline{ORBIT = ORBITAL} jako místo s 96\% pravděpodobností výskytu $e^-$

Matematicky vyjádřil vlnovou funkci $\Psi$ (psí)

Nositel Nobelovy ceny za fyziku 1933

\TabPositions{0em, 12em, 25em}
\subsection{Kvantová čísla}
\begin{description}
    \item[hlavní \textbf{n}] \tab 1-$\infty$(zatím 7) \tab udává \underline{energii} orbitu
    \item[vedlejší \textbf{l}] \tab 0-(n-1) \tab udává \underline{tvar} orbitu
    \item[magnetické \textbf{m}] \tab -l...0...+l \tab udává \underline{počet orbitalů} a jejich orientaci
    \item[spinové \textbf{s}] \tab $-\frac{1}{2}$ - $\frac{1}{2}$ \tab udává \underline{spin} $e^-$
\end{description}

\centering
\subsubsection{Slupky, energetické hladiny (dráhy)}
\begin{multicols}{2}
    $n = 1 \to K$

    $n = 2 \to L$
    
    $\vdots$
    
    $n = 3 \to M$
    
    $n = 4 \to N$

    $\vdots$ 
\end{multicols}

\centering
\subsubsection{Podslupky}
\begin{multicols}{2}
    $l = 0 \to s$

    $l = 1 \to p$
    
    $l = 2 \to d$
    
    $l = 3 \to f$
\end{multicols}


\subsubsection{Tvary orbitů}
\vspace{2em}
\raggedright
\TabPositions{0em, 20em}
$l = 0 \to$ tvar orbitu \textbf{s}: kulově symetrický
\tab
$\overbrace{\underbrace{1s}_{\cdot}}^{\text{\tiny{hlavní kv. \#, vedlejší \#}}}  \underbrace{2s}_{\circ} \hspace{3em} \underbrace{3s}_{\bigcirc}$

\vspace{2em}
\raggedright
$l = 1 \to $ tvar orbitu \textbf{p}: "ležatá osmička" \tab \hspace{2em} \huge{$\infty$}

\normalsize

$l = 2 \to $ tvar orbitu \textbf{d}: "čtyřlístek"

\vspace{2em}

$l = 3 \to $ tvar orbitu \textbf{f}: "velmi složitý tvar"


\raggedright

\subsection{Výstavbový princip}
\subsubsection[Znázornění orbitů a elektronů]{\underline{Znázornění orbitů a elektronů} v nich ($\downarrow \uparrow,~ \uparrow \uparrow,~ \downarrow\downarrow$)}
\begin{description}
    \TabPositions{0em, 12em}
    \vspace{1em}
    \item[a)] prostorovým tvarem: \tab s, p, d, f
    \vspace{2em}
    \item[b)] psaným symbolem: \tab \(
        \begin{array}{c}
            \qquad\qquad\;\; e^-\\
            \qquad \;\;\; \nearrow \\
            \;\;3d^7 \\
            \swarrow~\searrow \\
            \mbox{n=3} \qquad \mbox{l=2}
        \end{array}
    \)
    \hspace{6em}
    \(
        \begin{array}{c}
            \qquad\qquad\;\; e^-\\
            \qquad \;\;\; \nearrow \\
            \;\;4p^5 \\
            \swarrow~\searrow \\
            \mbox{n=4} \qquad \mbox{l=1}
        \end{array}
    \)
    \vspace{2em}
    \item[c)] rámečky: \tab  : 3 \fbox{$\downarrow\uparrow$}\fbox{$\downarrow\uparrow$}\fbox{$\downarrow$ }\fbox{$\downarrow$ }\fbox{$\downarrow$ } \hspace{3em} 4 \fbox{$\downarrow\uparrow$}\fbox{$\downarrow\uparrow$}\fbox{$\downarrow$ }
\end{description}

\newpage
\addcontentsline{toc}{subsubsection}{Příklad}
Příklad: Urči maximální počet $e^-$ ve slupce \textbf{N}

\smallskip
\TabPositions{0em, 7em}
N $\Rightarrow$ n=4 $\Rightarrow$\tab 0(s) $\Rightarrow$ m=0 (1 orbit)

    \tab 1(p) $\Rightarrow$ m=-1,0,1 (3 orbity)

    \tab 2(d) $\Rightarrow$ m=-2,-1,0,1,2 (5 orbity)

    \tab 3(f) $\Rightarrow$ m=-3,-2,-1,0,1,2,3 (7 orbity)

\medskip
Dohromady 16 orbitů * 2$e^-$ \textbf{= 32$e^-$}

...jelikož v každém orbitu mohou být 2 elektrony s opačným spinem (tzv. Pauliho vylučovací princip)

\bigskip
\fbox{\;\;} \underline{prázdný orbit = vakantní}

\subsubsection{Pravidla zaplňování orbitů}
\begin{enumerate}
    \item Pauliho vylučovací princip
    \item Hundovo pravidlo: Nejprve se zaplňují orbity jedním $e^- \; \Rightarrow$ nespárované $e^-$ mají stejný spin
    
    Příklad: $3d^7$: 3 \fbox{$\downarrow\textcolor{magenta}{\uparrow}$}\fbox{$\downarrow\textcolor{magenta}{\uparrow}$}\fbox{$\downarrow\textcolor{magenta}{.}$}\fbox{$\downarrow\textcolor{magenta}{.}$}\fbox{$\downarrow\textcolor{magenta}{.}$}
    
    Jedná se o tzv. \underline{DEGENEROVANÉ} orbity (mají stejné \underline{n} a \underline{l}, liší se v m) $\Rightarrow$
    
    $\Rightarrow$ orbity \textbf{s} nesjou degenerované, orbity \textbf{p} jsou 3x degenerované, orbity \textbf{d} 5x, \textbf{f} 7x
    \item Výstavbový princip: nejprve se zaplňují orbity s nízkou energií $\doteq$ v tomto pořadí:
    
    1s, 2s, 2p, 3s, 3p, 4s, 3d, 4p, 5s, 4d, 5p, 6s, 5d, 4f, 6p, 7s, 6d \dots
    \item Pravidlo \textbf{n+l}: Když je součet n+l stejný, zaplňují se provně orbity s nižší hodnotou \underline{n}.
\end{enumerate}

\subsubsection{Elektronové konfigurace podle výstavbového principu}
$\underline{\underline{_{13}}}Al$: $1s^2$, $2s^2$, $2p^6$, $3s^2$, $3p^1$ (součet $e^- = \underline{\underline{13}}$)

$\underline{\underline{_{26}}}Fe^-$: $1s^2$, $2s^2$, $2p^6$, $3s^2$, $3p^6$, $4s^2$, $3d^{\underline{7}}$ (součet $e^- = \underline{\underline{27}}$ - protože se jedná o záporný iont, má $e^-$ navíc!)

\subsubsection[Zápis se vzácným plynem]{Elektronové konfigurace podle předcházejícího vzácného(inertního) plynu - 8.hlps}
%\vspace{1em}
\(\underbrace{_{16}S\left[_{10}Ne\right]}_{16 - 10 = 6e^-}: \textcolor{red}{3}s^2, 3p^4 \longrightarrow 
    \mbox{\textcolor{red}{n} = zároveň \underline{\# periody} ve které se prvek nachází (S je ve 3. řádku per.tab.)}
\)

\vspace{1em}
Vždy se začíná orbitem \textbf{s} a pak další v pořadí \hyperlink{vystavbovyprincip}{výstavbového principu}

\vspace{2em}
$\underbrace{_{35}Br\left[_{18}Ar\right]}_{35 - 18 = 17e^-}$ : $4s^2, 3d^{10}, 4p^5$

\vspace{2em}
$\underbrace{_{53}I\left[_{36}Kr\right]}_{57 - 36 = 17e^-}$ : $5s^2, 4d^{10}, 5p^5$

\subsubsection{Elektronové konfigurace podle valenčních elektronů}
Valenční vrstva(svéra, též hladina) je poslední od jádra pro daný atom

\begin{description}
    \item[a)] \underline{Konfigurace základních (hlavních) prvků} (I.A - VIII.A)
\end{description}
        
\TabPositions{0em, 20em}
\section{Prvky}
\subsection{1. Hlavní podskupina - Alkalické kovy (tvoří hydroxidy)}
\textbf{H, Li, Na, K, Rb, Cs, Fr} (radioaktivní, 1940)

\textit{"Helenu Líbal Na Kolena Robot Cecil Franc"}
    \begin{itemize}
        \item s $\uparrow$ Z(protonové \#): $\, \uparrow$ \underline{m}, $\, \uparrow$r, $\, \downarrow$elektronegativita, $\, \downarrow t_t, \, \downarrow t_v$
        \item $ns^1$ \fbox{$\downarrow$} $\,\to\, "s^1$ prvky"
        
        \item vystupují jako elektropozitivní - malá IE, malá elektronegativita, vlevo v Beketovově řadě.
        \item oxidační \# ve sloučeninách = I $\,\to\,$ jsou redukčními činidly
    \end{itemize}
    \subsubsection{Vlastnosti}
        \begin{itemize}
            \item stříbrolesklé měkké kovy s malou hustotou (Li, Na, K jsou lehčí než voda)
        \end{itemize}
    \subsubsection{Výroba}
        elektrolýza tavenin halogenidů:
        \begin{itemize}
            \item $Na^+Cl^-\,\to\,$ na katodě$^{\textbf{-}}$
        \end{itemize}
    \subsubsection{Analytické důkazy - zbarvení plamene}
        Plamenové zkoušky
        \begin{itemize}
            \item Li - karmínově
            \item Na - žlutá
            \item K - fialová
        \end{itemize}
    Jsou \textbf{VELMI reaktivní} $\rightarrow$ výskyt \underline{jen ve sloučeninách}
    Musí se uchovávat v inertním prostředí N$_2$, petroleji...
    Sloučeniny:
    \begin{itemize}
        \item \textbf{NaCl - halit - sůl kamenná}
        \item KCl - sylvín
        \item Na$_2$CO$_3$ - soda
        \item NaHCO$_3$ - jedlá soda
        \item K$_2$CO$_3$ - potaš
        \item \textbf{sloučeniny s NO$_3$ - ledky} (výbuch v Bejrůtu 2020)
        \item NaNO$_3$ - ledek chilský
    \end{itemize}
    Výskyt v Zemské kůře Na: 2,4\%, K: 2,6\%
    
    \subsubsection{Reakce}
    \begin{enumerate}
        \item s H$_2\,\to\,$ HYDRIDY: \tab $2Na + H_2\,\to\, 2NaH$
        \item s O$_2\,\to\,$ OXIDY: \tab $4Li + O_2\,\to\, 2Li_2O$
            \newline s O$_2\,\to\,$ PEROXIDY:\tab $2Na + O_2\,\to\, Na_2O_2$
            \newline s O$_2\,\to\,$ HYPEROXIDY:\tab $K + O_2\,\to\, KO_2$
        \item s N$_2\,\to\,$ NITRIDY: \tab $6Li + N_2\,\to\, 2Li_3N$ (jen Li)
        \item s halogeny $\,\to\,$ HALOGENIDY: \tab $2Rb + Cl_2\,\to\, 2RbCl$
        \item s $\WATER \,\to\,$ HYDROXIDY (bouřlivě): \tab $2K + 2\WATER \,\to\, 2KOH + H_2$
    \end{enumerate}
    Jejich sloučeniny jsou často iontové, bazbarvé, rozpustné v $\WATER$

    \newpage
    \subsubsection{Hydroxidy (Louhy, "žíravé alkálie")}
        Leptají sklo, porcelán

        Výroba mýdel - zmýdelnění

        Jsou hydroskopické (přímají vzdušnou vlhkost):\[2\underline{NaOH} + CO_2\,\to\,\underline{Na}_2\underline{CO}_3 + \WATER \]
        \paragraph*{Výroba:}
            elektrolýza vodných $\odot$ halogenidů: ($H^+$ redukce na katodě$^-$, $Cl^-$ oxidace na anodě$^+$)
            \[\WATER \to H ^+ + OH^-\]
            \[NaCl \to Na^+ + Cl^-\]
            v $\odot$ zůstává $Na^+OH^-$ (\textbf{\underline{Na} se na katodě neredukuje $\Longleftarrow $ postavení v Beketovově řadě})
            Síla hydroxidů roste s jejich Z (protonové \#)
    \subsubsection{Význam}
        \begin{description}
            \item[Li] - výroba baterií (LiPo, LiFePo, LiIon), slouží při výrobě některých slitin
            \item[Na] - redukční činidlo: \(AlCl_3 + 3Na\,\to\,Al+3NaCl\)
            \item[K, Na] - biogenní prvky \begin{itemize}
                \item[-] sodíková "pumpa" (fungování nervového systému)
                \item[-] membránové potenciály - šíření signálu v nervech
            \end{itemize}
        \end{description}
    \subsubsection{Poznámka}
    $\odot$ NaCl = solanka
    \newline
    \newline
    Další dloučeniny:
    \begin{itemize}
        \item \textbf{$ Na _2 B _4 O _7 \, \cdot \, 10 \WATER$ (Borax)}
        \item $NaCN$
        \item $Na_2SiO_3$
        \item $K _2 Cr _2 O _7$
        \item $K O _2$ (hyperoxid draselný)
        \item $K _3 P O _4$
        \item $Na _2 SO _4 \, \cdot \, 10 \WATER$ (Glauberova sůl)
    \end{itemize}

\newpage
\subsection{2. Hlavní podskupina - Kovy alkalických zemin}
    \textbf{Be, Mg, Ca, Sr, Ba, Ra} (radioaktivní 1898 - manželé Marie a Peter Curie, smolinec)

    \textit{"Běžela Magda Canyonem, Srážela Banány Ramenem"}

    \begin{itemize}
        \item s $\uparrow$ Z(protonové \#): $\,\uparrow$ \underline{m}, $\,\uparrow$ r, $\,\downarrow$ elektronegativita
        \item $ns^2$ \fbox{$\uparrow \downarrow$} $\rightarrow \," s^2$ prvky"
        \item elektropozitivní \(X + \downarrow IE \,\to\,X^{II} + 2e^-\)
        \item vystupují jako elektropozitivní (+II) - malá IE, malá elektronegativita, vlevo v Beketovově řadě
    \end{itemize}

    \subsubsection{Vlastnosti}
        \begin{itemize}
            \item stříbrolesklé měkké kovy, kromě \underline{Be}
            \item Be se nejvíce podobá Al, \textbf{má amfoterní charakter!}
        \end{itemize}

    \subsubsection{Analytické důkazy - zbarvení plamene}
        Plamenové zkoušky
        \begin{itemize}
            \item Ca - cihlová
            \item Sr - karmínová
            \item Ba - žlutozelená
            \item Mg - silná záře (jako při řezání autogenem):  $2Mg + O _2 \,\to\, 2MgO$
        \end{itemize}
    Jsou reaktivní méně než prvky 1.hlps $\Rightarrow$ výskyt ve sloučeninách:
        \begin{itemize}
            \item $CaCO_3$ - vápenec (aragonit, sintr, mramor, travertin. kalcit...)
            \item $CaF_2$ - fluorit = kazivec
            \item $BaSO _4$ - barit
            \item $MgCO_3$ - magnezit
            \item $CaCO_3 \, \cdot \, MgCO_3$ - dolomit
            \item $CaSO_4 \, \cdot \, 2\WATER$ - sádrovec (sádra: $CaSO_4 \,\cdot\, \frac{1}{2}\WATER$)
        \end{itemize}

    \subsubsection{Výroba}
        \begin{description}
            \item[a)] \underline{elektrolýza tavenin} jejich \underline{halogenidů}: \textbf{\(Ca^{2+}Cl_2\)} ($Ca^{2+}$ redukce na katodě$^-$)
            \item[b)] \underline{aluminotermie}(Al je redukční činidlo): \(3BeO + Al \,\to\, 3Be + Al_2 O_3\)
        \end{description}
    \subsubsection{Reakce}
    \begin{enumerate}
        \item s $H_2 \,\to$ HYDRIDY: \tab $Ca + H_2 \,\to\, CaH_2$
        \item s $O_2 \,\to$ OXIDY: \tab $2Ba + O_2 \,\to\, 2BaO$
            \newline s $O_2 \,\to\,$ PEROXIDY: \tab $Ba + O_2 \,\to\, BaO_2$ (peroxid barnatý!)
        \item s $N_2 \,\to\,$ NITRIDY: \tab $3Sr + N_2 \,\to\, Sr_3 N_2$
        \item s $\WATER \,\to\,$ HYDROXIDY: \tab $Ca+2\WATER \,\to\, Ca \left( OH \right) _2 + H_2$ (exotermická reakce)
            \newline \tab\tab $Ba + 2\WATER \,\to\, \underbrace{Ba \left( OH \right) _2}_\text{barytová voda} + H_2$ 
    \end{enumerate}
    Sloučeniny Ca (stavebnictví)
    \[\underbrace{CaCO_3}_\text{vápenec} \,\overrightarrow{\, _{800^\circ C} \,} \, \underbrace{CaO}_\text{pálené vápno} + CO_2\]
    \newline
    \[CaO + 2\WATER \,\to\, \underbrace{Ca \left(OH\right)_2}_\text{hašené vápno}\]
    \newline
    \[Ca\left(OH\right)_2 + \underbrace{CO_2 \downarrow}_{ze~vzduchu} ~\to~ CaCO_3 + \WATER\] ...princip tvrdnutí malty
    \newline \newline
    \underline{Podstata krasových jevů:} Uhličitany jsou ve vodě nerozpustné, ale v přítomnosti $CO_2$ (ze vzduchu) se rozpouštějí:
    \[CaCO_3 + CO_2 + \WATER ~\rightleftharpoons ~ Ca \left( HCO_3 \right)_2 \]
    Zpětná rekristalizace na $CaCO_3$ = minerál \underline{sintr} - krápníky
    \begin{description}
        \item[a)] stalagnit - $\bigwedge$
        \item[b)] stalagtit - $\bigvee$
        \item[c)] stalagnát - spojený \tiny{(..nenašel jsem vhodný znak x, btw proč všichni Češi znají krápníky, ale když se jich zeptáš na prvního prezidenta tak budou tupě čumět.)}
    \end{description}

    \subsubsection{Význam}
    \begin{description}
        \item[Ca, Mg] - biogenní prvky
        \item[Ca] - kosti, zuby
        \item[Mg] \textbf{- součást molekuly chlorofilu}
        \item[Be] - lehký tvrdý kov (o 30\% lehční než Al), slitiny se používají pro výrobu nástrojů i raket, sloučeniny jsou toxické
    \end{description}

    \subsubsection{Poznámka}
    Minerál beryl [$3BeO \cdot Al_2O_3 \cdot 6SiO_2$]

    - oxidy smaragd(zelený) a akvamarín(modrý)

\subsection{3. Hlavní podskupina - $p^1$ prvky}
\textbf{B, Al, Ga, In, Th}
"Byl Ale Gagarin Indická Tlama", "Běžela Alena Gálií, Indiáni Táhli jí"

\underbrace{B}_{nekov}, \underbrace{Al, Ga, In, Tl}_{kovy}

s $\uparrow$Z: $\uparrow$m, $\uparrow$r, kovový charakter, $\downarrow$elektronegativita

Valenční elektrony: 
\underbrace{$ns^2$}_{\fbox{$\uparrow\downarrow$}}, \underbrace{$np^1$}_{\fbox{$\downarrow\phantom{\uparrow}$}\fbox{\phantom{$\downarrow\uparrow$}}\fbox{\phantom{$\downarrow\uparrow$}}}
$\; \rightarrow$ \underline{hl}.(nejčastější oxidační \# = III)

\subsection{B (bor, borum, borine (en))}
Vázaný ve sloučeninách, nekovový prvek, málo reaktivní, využívá se jako moderátor v jaderných reaktorech (například v Jaderné elektrárně Temelín)

\subsubsection{Minerály}: \underline{borax} = $Na_2B_4O_7 \; \cdot \; 10H_2O$

v analytiké chemii "boraxová perlička" - při 900\˚C $\rightarrow$ sklovitá hmota,
která se v přítomnosti různých iontů zabarvuje:
\begin{itemize}
    \item $Co^{2+}$ ... modrá
    \item $Mn^{2+}$ ... fialová
    \item $Cr^{3+}$ ... zelenáf
\end{itemize}

\subsubsection{Příprava:}
\[B_2O_3 \; + \; 3Mg \; \longrightarrow \; 2B \; + \; 3MgO\]

\[B_2O_3 \; + \; 2Al \; \longrightarrow \; 2B \; + \; Al_2O_3\] aluminotermie

\subsubsection{Sloučeniny}
Borany = borovodíky (obecný vzorec $B_nH_{2n+2}$)

\[\underbrace{Mg_3B_2}_\text{borid hořečnatý} \; + \; 6H_2O \; \longrightarrow \; 3Mg\left(OH\right)_2 \; + \; \underbrace{B_2H_6}_\text{diboran - plyn}\]

$H_3BO_3$ - kys trihydrogenboritá (ortoboritá), její 3\%$\odot$ je "borová voda"

$H_3BO_3 \; \overrightarrow{\text{var, -$H_2O$}} \; HBO_2$ (kyselina hydrogen boritá)

BN - nitrid borum

$B_4C$ - karbid boru - brusný materiál, velmi tvrdá černá krystalická látka.
Používá se na výrobu neprůstřelných oděvů, 
    \section{Radioaktivita}
\underline{Uranové paprsky} - objev \underline{Becquerel} (1896)
$\rightarrow$ ozáření fotografické desky (kámen \textbf{smolinec} z Jáchymova)

\vspace{1em}

\underline{Marie Curie Sklodowská} + manžel \underline{Pierre Curie} - objev $_{84}Po$ (polonia) a $_{88}Ra (radia)$

$\rightarrow$ paprsek $=$ \underline{radioaktivita} - V roce 1903 udělení Nobelovy ceny pro Marii, Piera a Becquerela

K maturitě je třeba znát stručný životopis rodiny Curie a Sklodowských.

\subsection{Termíny}

\begin{itemize}
    \item \underline{IZOTOPY}: Stejné Z(protonové \#), liší se počtem neutronů \begin{itemize}
        \item př. $^1_1H$ (vodík, protium), $^2_1H$ (deuterium), $^3_1H$ (tritium)
        \item př. $^{12}_6C$, $^{13}_6C$, $^{14}_6C$ (radioaktivní) $\Rightarrow$ radiouklíkové datování (stanovení stáří organických materiálů)
        \item př. $^{235}_{92}U$, $^{237}_{92}U$, $^{238}_{92}U$ atd.
        \end{itemize}
    \item \underline{IZOBARY}: Jiné Z, stejná A(nukleonové \#) - př. $^{40}_{20}Ca$ a $^{40}_{19}K$
    \item \underline{IZOTONY}: Stejný počet neutronů - př. $^{12}_5B$ a $^{13}_6C$ (oba mají 7$^1_0n$)
\end{itemize}

\subsection{Druhy záření}

$^4_2\alpha \; = \; ^4_2He$ - alfa záření se šíří cca $\frac{1}{10}c$ (rychlosti světla), zachytí se i listem papíru

\vspace{2em}

$\beta$: \begin{itemize}
    \item $\beta^{-} \; = \; ^0_{-1}e$ (elektron) - šíří se cca $\frac{9}{10}c$, záchyt kovovými fóliemi (alobal)
    \item $\beta^{+} \; = \; ^0_{+1}e$ (pozitron)
\end{itemize}

\vspace{2em}

$\gamma$ (gama) $=$ elektromagnetické záření - proud fotonů, rychlost světla, záchyt olověnými deskami, betonem, \underline{zhoubné}

\subsection{Poločas rozpadu T}
Lepší název je \underline{Poločas přeměny}, jelikož né každá přeměna jádra musí být rozpadem (může se jednat třeba o emisy $\gamma$ záření)

$T_{\frac{1}{2}} \; = \; \frac{ln2}{\lambda}$, konstanta určující dobu, za kterou se rozpadne $\frac{1}{2}$ jader daného prvku $\Rightarrow$ exponenciální graf.
\textbf{$T_{\frac{1}{2}}$} jednodlivých prvků zle najít v tabulkách: 
\begin{itemize}
    \item př. $^{14}_6C \rightarrow T \doteq 5.7$tisíce let
    \item př. $^{208}_{84}Po \rightarrow T \doteq 2.9$roku
    \item př. $^{209}_{84}Po \rightarrow T \doteq 103$let
    \item př. $^{210}_{84}Po \rightarrow T \doteq 138.4$dní
\end{itemize}

\newpage
\subsubsection{Úloha o poločasu rozpadu}

Víme, že při vzniku vzorku obsahoval 1 atom $^{14}_6C$ na $10^{12}$ atomů uhlíku $^{12}_6C$ (jelikož tento poměr je v organickém materiálu v atmosféře dlouhodobě stálý)

Při posledním měření bylo ve vzorku nameřen poměř $1 : 1.414*10^{12}$ = $^{14}C : ^{12}C$.

Poločas rozpadu uhlíku $^{14}C$ je 5730let. Jak starý je vzorek?

\rule{5em}{1px}

\begin{itemize}
    \item Původní koncentrace $^{14}C$ ... $c_p = (10^{12})^{-1} = 10^{-12}$
    \item Naměřená koncentrace $^{14}C$ ... $c_m = (1.414 \times 10^{12})^{-1} \doteq 7.07 \times 10^{-13}$
    \item Poločas rozpadu $T_{\frac{1}{2}} = 5730$let
    \item Uplynulá doba od smrti vzorku ... $t = ?$
\end{itemize}

\begin{align*}
    c_{m} &= c_{p} \times \left(\frac{1}{2}\right)^{t \; \div \;  T_{\frac{1}{2}}} \\
    7.07 \times 10^{-13} &= 10^{-12} \times \left(\frac{1}{2}\right)^{t \; \div \;  5730} \\
    \frac{7.07 \times 10^{-13}}{10^{-12}} &= \left(\frac{1}{2}\right)^{t \; \div \;  5730} \\
    \log_{\frac{1}{2}}\left(\frac{7.07 \times 10^{-13}}{10^{-12}}\right) &= t \; \div \;  5730 \\
    t &= \log_{\frac{1}{2}}\left(\frac{7.07 \times 10^{-13}}{10^{-12}}\right) \times 5730 \\
    t &\doteq 2866 let
\end{align*}

\TabPositions{0em, 15em, 30em}
\subsection{Rozpadové řady}
Přirozené:
\begin{enumerate}
    \item URANOVÁ: \tab $^{238}_{92}U \; \cdots \longrightarrow \; ^{206}_{82}Pb$ \tab A = 4n + 2
    \item THORIOVÁ: \tab $^{232}_{90}Th \cdots \longrightarrow \; ^{208}_{82}Pb$ \tab A = 4n
    \item AKTINOURANOVÁ: \tab $^{235}_{92}U \; \cdots \longrightarrow \; ^{207}_{82}Pb$ \tab A = 4n + 3 
\end{enumerate}

Umělá:
\begin{enumerate}
    \setcounter{enumi}{3}
    \item NEPTUNIOVÁ: \tab $^{237}_{93}Np \cdots \longrightarrow \; ^{205}_{81}Tl$ \tab A = 4n + 1
\end{enumerate}

\subsubsection{Příkad}
Do které řady patří $\; ^{234}_{92}U$ ?

\begin{align*}
    234&\div 4 = 58 \\
    34& \\
    \underline{2}& \; \longleftarrow \mbox{4n + \underline{2}} \; \Rightarrow \mbox{Uranová řada}\\
\end{align*}

Uran234 patří do uranové řady, protože zbytek po dělení jeho A (nukleonového \#) čtyřmi je 2.

\subsection{Umělá radioaktivita}
dcera \underline{Irene Curie} + manžel \underline{F.J.Curie}
Vznik umělých radioizotopů (medicína, konzervace potravin, sterilizace materiálů...)

\[^{27}_{13}Al + \; ^4_2\alpha \; \longrightarrow \; ^{30}_{15}P + \; ^1_0n\]
\vspace{1em}
\[^{238}_{92} +  \; ^1_0n \; \longrightarrow \; ^{237}_{92}U + \; 2^1_0n\]
\vspace{0.5em}

\textbf{Součet čísel na obou stranách se MUSÍ rovnat}

\rule{44em}{1px}

\begin{multicols}{3}
    \vspace{0.5em}
    proton: $^1_1p$
    
    \vspace{0.5em}
    pozitron: $^0_1e$
    
    \vspace{0.5em}
    $^4_2\alpha = ^4_2He$
    
    \vspace{0.5em}
    neutron: $^1_0n$
    
    deuterium: $^2_1D = ^2_1H$
    
    $\beta^{-} = ^0_{-1}e$
    
    elektron: $^0_{-1}e$
    
    tritium: $^3_1T = ^3_1H$
    
    $\beta^{+} = ^0_1e = pozitron$
\end{multicols}

\subsection{Posuvové zákoky}
Vytváří-li prvek:
\begin{itemize}
    \item $^4_2\alpha \; \Longrightarrow \;$ A - 4, Z - 2
    \item $\beta^- \; \Longrightarrow \;$ A, \;Z + 1
    \item $\beta^+ \; \Longrightarrow \;$ A, \;Z - 1
\end{itemize}
\textbf{Příklad:} Napiš produkty přeměn:
\begin{enumerate}
    \item rozpadem $\alpha$: \tab \(^{226}_{88}Ra \; \rightarrow \; ^4_2\alpha \; + \; ^{222}_{86}Rn\)
    \item rozpadem $\beta^-$: \tab \(^{32}_{15}P \; \rightarrow \; ^0_{-1}e + ^{32}_{16}X\)
    \item rozpadem $\beta^+$: \tab \(^{11}_{6}C \; \rightarrow \; ^0_{1}e + ^{11}_{5}X\)
\end{enumerate}

\subsection{Jaderné reakce}
Musí být dodržen:
\begin{itemize}
    \item \textbf{Zákon zachování energie}
    \item zákon zachování hybnosti
    \item zachování elektrického náboje
    \item zachování počtu nukleonů
\end{itemize}

Dělení: transmutace, štepení, fůze

\subsubsection{Transmutace}
Reakce při nichž se mění jádro prvku na jiné, které se liší maximálně o \textbf{2 v Z} a o \textbf{4 v A}
Příklady:
\[^{209}_{83}Bi \; + \; ^4_2\alpha \; \longrightarrow \; ^{211}_{85}Az \; + \; 2^1_0n\]
\vspace{0.2em}
\[^{41}_{19}K \; + \; ^2_1D \; \longrightarrow \; ^{42}_{19}K \; + \; ^1_1p\]
\vspace{0.2em}
\[^{10}_{5}B \; + \; ^1_0n \; \longrightarrow \; ^{7}_{3}Li \; + \; ^4_2\alpha\]

\subsubsection{Stěpení jader}
Reakce při nichž se štěpí těžká jádra na (obvykle) 2 středně těžká jádra + neutron(y) + velké množství energie (v MeV - megaelektronvolt)
Příklady:
\[^{235}_{92}U \; + \; ^1_0n \; \longrightarrow \; _{56}Ba \; + \; _{36}Kr \; + \; 3^1_0n\]
\vspace{0.1em}
\[^{235}_{92}U \; + \; ^1_0n \; \longrightarrow \; _{54}Xe \; + \; _{38}Sr \; + \; 2^1_0n\]
Jádra se štěpí s určitou pravděpodobností

\subsubsection{Řetězová reakce}
Potvrzeno na \underline{jaře 1939}

Ze štěpení jádra atomem se uvolňují další neutrony, které štěpí další atomy atd.

Jako palivo se běžně používá izotop $^{235}_{92}U$, občas také $^{239}_{94}Pu$ (plutonium)

Řetězová štěpná reakce je kromě atomových elektráren také podstatou jaderné bomby.

\vspace{1em}

\underline{ENRICO FERMI} 2.12.1942 poprvé uskutečnil \underline{řízenou řetězovou reakci} (v jaderném reaktoru na hřišti univezity v Chicagu).
Fermi je nositelem Nobelovy ceny za z roku 1938 za přípravu 1. transuranu pPrvku s vyšší protonovým číslem než uran) \textbf{Z = 93 $\rightarrow$ Np}

\subsubsection{Projekt Manhattan}
%TODO: Dopsat
"Otec" atomové bomby: Robert Oppenheimer

Dále na ní pracovali například: Fermi, Bohr, Einstein, Feinman, Meitner (žena), Heisenberg, Landau, Kurčatov, Gamow 

První užití jaderné zbraně: červenec 1945 Hirošima, poté Nagasaki

\vspace{1em}
\underline{Termíny}:
\begin{itemize}
    \item \underline{obohacování uranu} izotopem $^{235}_{92}U$ (mezinárodní dohody zakazují nad 5\%)
    \item \underline{kritické množství} (critical mass) $^{235}_{92}U$ je zhruba 44.5kg (koule o průměru 16.8cm)
    \item \underline{atomový reaktor}
    \item úložiště jaderného odpadu
    \item \underline{moderátor} v jaderné elektrárně: snižuje rychlost volných neutronů: grafit, parafin, $D_2O$, sloučeny boru
    \item Těžká voda = $D_2O$ - voda obsahující izotop vodíku Deuterium ($^2_1D$) - má jiné fyzikální i chemické vlastnosti. M = 20, jiné body tání a mrznutí... Organizmy v ní nepřežívají
\end{itemize}

\underline{Jaderné elektrárny}:
\begin{itemize}
    \item Jaderná elektrárna Dukovany (ČR, v provozu od 1985)
    \item Jaderná elektrárna Temelín (ČR, v provozu od 2002)
    
    \item Jaderná elektrárna Chornobyl (Černobyl) - na Ukrajině, velká havárie 26.dubna 1986 - výbuch 4. jaderného bloku během experimentů s jeho odstavováním.
    Poblíž (3km) leží město Pripjať
\end{itemize}

\subsubsection{Jaderná fůze}
Též \underline{jaderná syntéza}, termonukleární reakce

Skládání jader na jádra těžší.

Samovolně probíhá ve hvězdách, například ve Slunci (zatím na He).

Uvolňuje se \underline{obrovské} množství energie. Spývají jádra bez elektronového obalu

\vspace{1em}
Příklady:
\[^2_1D \; + \; ^2_1D \; \longrightarrow \; ^3_2He \; + \; ^1_0n\]
\[^7_3Li \; + \; ^1_1H \; \longrightarrow \; 2^4_2He\]
\[^7_3Li \; + \; ^2_1D \; \longrightarrow \; 2^4_2He \; + \; ^1_0n\]
\[^2_1D \; + \; ^3_1T \; \longrightarrow \; ^4_2He \; + \; ^1_0n\]

\vspace{1em}
Reaktory jsou v US a na jihu Francie. Zatím neumíme fůzy řídit.

\vspace{1em}

Výhody: dostatek surovin (D,T), není odpad - jen netečné He, není radioaktivní (jen $^3_1Y$), bezpečnost - zdá se, že se jedná o ideální zdroj energie.
    \section{Chemická vazba}
Atomy se k sobě přibližují na optimální vzdálenost až se překryjí jejich valenční orbity a vznikne vazebný elektronový pár.

\underline{Při vzniku} chemické vazby \underline{se energie uvolňuje} $\rightarrow$ Stabilita

\subsection{Znázornění chemických vazeb}
\TabPositions{0em, 15em, 30em}
\begin{enumerate}
    \item prostorovým tvarem orbitů
    \item valenční čárkou \tab H + H $\longrightarrow$ $H_2$ \tab H---H
    \item rámečky \tab H + H $\longrightarrow$ $H_2$ \tab \fbox{$\rightarrow$}---\fbox{$\leftarrow$}
\end{enumerate}

\subsection{Kovalentní vazba} - společně sdílejí $e^-$

\vspace{1em}
\underline{ELEKTRONEGATIVITA} je schopnost atomu přitahovat vazebné $e^-$.

V periodách roste, ve sloupcích klesá

Jedná se o bezrozměrné číslo (nemá jednotku)

\subsubsection{Nepolárně kovalentní}
Rozdíl elektronegativit mezi vázanými atomy od 0 - 0,4

\subsubsection{Polárně kovalentní}
Rozdíl elektronegativit mezi vázanými atomy od 0,4 - 1,7

Například $H^{\delta+}$---$Cl^{\delta-}$ $\longleftarrow \; \delta$=delta, částečný, parciální náboj

\subsubsection{Iontová vazba}
Rozdíl elektronegativit mezi vázanými atomy \>1,7

Příklad: \(KBr \; \longrightarrow \; K^+ \; + \; Br^-\)
%% save the original value of the \textheight parameter:
\newlength\origheight
\setlength\origheight{\textheight}

\section{Přehledy}
    \begin{landscape}
\subsection{Vitaminy}
    \begin{longtable}{| m{7em} | m{5em} | m{5em} | m{8em}<{\RaggedRight} | m{15em}<{\RaggedRight} | m{15em}<{\RaggedRight} | m{10em}<{\RaggedRight} |}
        %\hline
        %\multicolumn{7}{| c |}{Přehled vitaminů}\\
        \hline
        Název & Skupina & Správná denní dávka & Zdroj & Význam & Projevy nedostatku & Poznámka \\
        \hline
        A (retinol) & tetraterpen & 1.8-2mg & mléčný tuk, vaječný žloutek, játra, rybí tuk i maso, barevná zelenina & zajišťuje vidění, tvoří oční purpur, podílí se na tvoření bílkovin v kůži a ve sliznicích & šeroslepost, rohovatění kůže a sliznic, ucpávání vývodů žláz, postižení skloviny i zuboviny & nebezpečí hypervitaminózy z předávkování - bolest hlavy, koliky, průjmy \\
        \hline
        B (thiamin)& heterocykl & 1.5mg & obiloviny(zejména klíčky), kvasnice, játra, vepřové maso & zasahuje především do metabolismu cukrů, zejména v centrálním nervstvu a ve svalech; podporuje činnost trávicího ústrojí & zvýšená únavnost, sklony ke křečím svalstva, srdeční poruchy, trávicí poruchy, dispozice k zánětům nervů až onemocnění beri-beri & \\
        \hline
        $B_1$ (riboflavin) & & 1.8mg & mléko, maso, kvasnice & jako účinná složka tzv. žlutého dýchacího fermentu je v každé buňce, kde se účastní oxidace živin & zardělost a palčivost jazyka, zduření rtů, bolavé koutky, poruchy sliznice hltanu a hrtanu & v 1litry mléka je okolo 1mg \\
        \hline
        $B_3$ (kys. pantotenová) & deriv. kys. máselné & 7-10mg & játra, kvasnice, hrách, maso, mléko, vejce & účast v oxidoreduktázách a umožňuje syntézu bílkovin+ jako koenzym A má centrální postavení v metabolizmu & různé degenerace; u člověka pálení chodidel & je ve všech tkáních \\
        \hline
        $B_6$ (pyridoxin) & & 2mg & kvasnice, obilné klíčky, mléko, luštěniny & podporuje účinek vitaminů $B_1 ~a~ B_3$ & pomalé hojení zánětů, zhoršení regenerace sliznic & \\
        \hline
        $B_{12}$ (kobalamin)& & 0.001mg & játra, maso, činností bakterií vznik ve střevech & nutný pro udržení normální krvetvorby & "zhoubná" chudokrevnost & ke vstřebávání vitaminu $B_{12}$ je nutná přítomnost tzv. vnitřního faktoru \\
    \endfirsthead
        \hline
        Kys. nikotinová & heterocykl & 15-20mg & játra, ledviny, maso, kvasnice, houby & klíčová pro syntézu ribonukleových kyselin a bílkovin & záněty kůže, celková sešlost, poškození mozku& \\
        \hline
        Kys. listová & heterocykl & 0.5-1mg & listová zelenina & zasahuje do metabolismu aminokyselin, je nutná pro tvorbu červených krvinek & chudokrevnost & \\
        \hline
        C (kys. askorbová) & Sacharid deriv. & 50-70mg & syrové ovoce a zelenina & katalyzuje oxidaci živin, udržuje dobrý stav vaziva a chrupavek, podporuje tvorbu protilátek & únava, snížená odolnost proti nakažlivým nemocem, krvácení, vypadávání zubů; při avitaminóze vzniká smrtelné onemocnění kurděje & předávkování C vitaminu může být i zdravý škodlivé \\
        \hline
        D (vit. antirachitický) & steroid & 400m.j. & rybí tuk, vzinká po ozáření UV v malém množství i v kůži & podílí se na řízení metabolismu Ca a P v těle & ztrácí-li organismus Ca a P, snaží se jej nahradit z kostí, za vývoje vzniká křivice, v dospělosti měknutí kostí, rachitis & hypervitaminóza D vede k ukládání Ca v ledvinách, srdci, stěnách cév a může ohrozit život \\
        \hline
        E (tokoferol) & deriv. tokolu & 5-20mg & obilné klíčky & podporuje činnost pohlavních žláz a správný průběh těhotenství & některé gestační poruchy & \\
        \hline
        H (Biotin) & heterocykl & 0.15-0.3mg & kvasnice, játra, ledviny, biosyntéza ve střevech & je ve všech živočišných buňkách, podporuje jejich růst a dělení & záněty kůže, atrofie papil jazyka, unavenost, deprese, svalové bolesti, nechutenství & \\
        \hline
        K (vit. antihemoragický) & deriv. naftochinonu & 1mg & listové zeleniny, kvasnice, v tlustém střevě je tvořen činností mikroorganismů & oxidoreduktáza, tvorba protisrážlivé látky protrombinu & krvácení do tkání a tělesných dutin, krvácení do mozku může zapříčinit smrt & \\
        \hline
        %\caption{Přehled vitaminů}
    \end{longtable}
\end{landscape}

\end{document}