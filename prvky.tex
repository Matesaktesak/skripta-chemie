    
\TabPositions{0em, 20em}
\section{Prvky}
\subsection{1. Hlavní podskupina - Alkalické kovy (tvoří hydroxidy)}
    \textbf{H, Li, Na, K, Rb, Cs, Fr} (radioaktivní, 1940)

    \textit{"Helenu Líbal Na Kolena Robot Cecil Franc"}
    \begin{itemize}
        \item s $\uparrow$ Z(protonové \#): $\, \uparrow$ \underline{m}, $\, \uparrow$r, $\, \downarrow$elektronegativita, $\, \downarrow t_t, \, \downarrow t_v$
        \item $ns^1$ \fbox{$\downarrow$} $\,\to\, "s^1$ prvky"
        
        \item vystupují jako elektropozitivní - malá IE, malá elektronegativita, vlevo v Beketovově řadě.
        \item oxidační \# ve sloučeninách = I $\,\to\,$ jsou redukčními činidly
    \end{itemize}
    \subsubsection*{Vlastnosti}
        \begin{itemize}
            \item stříbrolesklé měkké kovy s malou hustotou (Li, Na, K jsou lehčí než voda)
        \end{itemize}
    \subsubsection*{Výroba}
        elektrolýza tavenin halogenidů:
        \begin{itemize}
            \item $Na^+Cl^-\,\to\,$ na katodě$^{\textbf{-}}$
        \end{itemize}
    \subsubsection{Analytické důkazy - zbarvení plamene}
        Plamenové zkoušky
        \begin{itemize}
            \item Li - karmínově
            \item Na - žlutá
            \item K - fialová
        \end{itemize}
    Jsou \textbf{VELMI reaktivní} $\rightarrow$ výskyt \underline{jen ve sloučeninách}
    Musí se uchovávat v inertním prostředí N$_2$, petroleji...
    Sloučeniny:
    \begin{itemize}
        \item \textbf{NaCl - halit - sůl kamenná}
        \item KCl - sylvín
        \item Na$_2$CO$_3$ - soda
        \item NaHCO$_3$ - jedlá soda
        \item K$_2$CO$_3$ - potaš
        \item \textbf{sloučeniny s NO$_3$ - ledky} (výbuch v Bejrůtu 2020)
        \item NaNO$_3$ - ledek chilský
    \end{itemize}
    Výskyt v Zemské kůře Na: 2,4\%, K: 2,6\%
    
    \subsubsection{Reakce}
    \begin{enumerate}
        \item s H$_2\,\to\,$ HYDRIDY: \tab $2Na + H_2\,\to\, 2NaH$
        \item s O$_2\,\to\,$ OXIDY: \tab $4Li + O_2\,\to\, 2Li_2O$
            \newline s O$_2\,\to\,$ PEROXIDY:\tab $2Na + O_2\,\to\, Na_2O_2$
            \newline s O$_2\,\to\,$ HYPEROXIDY:\tab $K + O_2\,\to\, KO_2$
        \item s N$_2\,\to\,$ NITRIDY: \tab $6Li + N_2\,\to\, 2Li_3N$ (jen Li)
        \item s halogeny $\,\to\,$ HALOGENIDY: \tab $2Rb + Cl_2\,\to\, 2RbCl$
        \item s $\WATER \,\to\,$ HYDROXIDY (bouřlivě): \tab $2K + 2\WATER \,\to\, 2KOH + H_2$
    \end{enumerate}
    Jejich sloučeniny jsou často iontové, bazbarvé, rozpustné v $\WATER$

    \newpage
    \subsubsection{Hydroxidy (Louhy, "žíravé alkálie")}
        Leptají sklo, porcelán

        Výroba mýdel - zmýdelnění

        Jsou hydroskopické (přímají vzdušnou vlhkost):\[2\underline{NaOH} + CO_2\,\to\,\underline{Na}_2\underline{CO}_3 + \WATER \]
        \paragraph*{Výroba:}
            elektrolýza vodných $\odot$ halogenidů: ($H^+$ redukce na katodě$^-$, $Cl^-$ oxidace na anodě$^+$)
            \[\WATER \to H ^+ + OH^-\]
            \[NaCl \to Na^+ + Cl^-\]
            v $\odot$ zůstává $Na^+OH^-$ (\textbf{\underline{Na} se na katodě neredukuje $\Longleftarrow $ postavení v Beketovově řadě})
            Síla hydroxidů roste s jejich Z (protonové \#)
    \subsubsection{Význam}
        \begin{description}
            \item[Li] - výroba baterií (LiPo, LiFePo, LiIon), slouží při výrobě některých slitin
            \item[Na] - redukční činidlo: \(AlCl_3 + 3Na\,\to\,Al+3NaCl\)
            \item[K, Na] - biogenní prvky \begin{itemize}
                \item[-] sodíková "pumpa" (fungování nervového systému)
                \item[-] membránové potenciály - šíření signálu v nervech
            \end{itemize}
        \end{description}
    \subsubsection*{Poznámka}
    $\odot$ NaCl = solanka
    \newline
    \newline
    Další dloučeniny:
    \begin{itemize}
        \item \textbf{$ Na _2 B _4 O _7 \, \cdot \, 10 \WATER$ (Borax)}
        \item $NaCN$
        \item $Na_2SiO_3$
        \item $K _2 Cr _2 O _7$
        \item $K O _2$ (hyperoxid draselný)
        \item $K _3 P O _4$
        \item $Na _2 SO _4 \, \cdot \, 10 \WATER$ (Glauberova sůl)
    \end{itemize}

\newpage
\subsection{2. Hlavní podskupina - Kovy alkalických zemin}
    \textbf{Be, Mg, Ca, Sr, Ba, Ra} (radioaktivní 1898 - manželé Marie a Peter Curie, smolinec)

    \textit{"Běžela Magda Canyonem, Srážela Banány Ramenem"}

    \begin{itemize}
        \item s $\uparrow$ Z(protonové \#): $\,\uparrow$ \underline{m}, $\,\uparrow$ r, $\,\downarrow$ elektronegativita
        \item $ns^2$ \fbox{$\uparrow \downarrow$} $\rightarrow \," s^2$ prvky"
        \item elektropozitivní \(X + \downarrow IE \,\to\,X^{II} + 2e^-\)
        \item vystupují jako elektropozitivní (+II) - malá IE, malá elektronegativita, vlevo v Beketovově řadě
    \end{itemize}

    \subsubsection*{Vlastnosti}
        \begin{itemize}
            \item stříbrolesklé měkké kovy, kromě \underline{Be}
            \item Be se nejvíce podobá Al, \textbf{má amfoterní charakter!}
        \end{itemize}

    \subsubsection*{Analytické důkazy - zbarvení plamene}
        Plamenové zkoušky
        \begin{itemize}
            \item Ca - cihlová
            \item Sr - karmínová
            \item Ba - žlutozelená
            \item Mg - silná záře (jako při řezání autogenem):  $2Mg + O _2 \,\to\, 2MgO$
        \end{itemize}
    Jsou reaktivní méně než prvky 1.hlps $\Rightarrow$ výskyt ve sloučeninách:
        \begin{itemize}
            \item $CaCO_3$ - vápenec (aragonit, sintr, mramor, travertin. kalcit...)
            \item $CaF_2$ - fluorit = kazivec
            \item $BaSO _4$ - barit
            \item $MgCO_3$ - magnezit
            \item $CaCO_3 \, \cdot \, MgCO_3$ - dolomit
            \item $CaSO_4 \, \cdot \, 2\WATER$ - sádrovec (sádra: $CaSO_4 \,\cdot\, \frac{1}{2}\WATER$)
        \end{itemize}

    \subsubsection*{Výroba}
        \begin{description}
            \item[a)] \underline{elektrolýza tavenin} jejich \underline{halogenidů}: \textbf{\(Ca^{2+}Cl_2\)} ($Ca^{2+}$ redukce na katodě$^-$)
            \item[b)] \underline{aluminotermie}(Al je redukční činidlo): \(3BeO + Al \,\to\, 3Be + Al_2 O_3\)
        \end{description}
    \subsubsection{Reakce}
    \begin{enumerate}
        \item s $H_2 \,\to$ HYDRIDY: \tab $Ca + H_2 \,\to\, CaH_2$
        \item s $O_2 \,\to$ OXIDY: \tab $2Ba + O_2 \,\to\, 2BaO$
            \newline s $O_2 \,\to\,$ PEROXIDY: \tab $Ba + O_2 \,\to\, BaO_2$ (peroxid barnatý!)
        \item s $N_2 \,\to\,$ NITRIDY: \tab $3Sr + N_2 \,\to\, Sr_3 N_2$
        \item s $\WATER \,\to\,$ HYDROXIDY: \tab $Ca+2\WATER \,\to\, Ca \left( OH \right) _2 + H_2$ (exotermická reakce)
            \newline \tab\tab $Ba + 2\WATER \,\to\, \underbrace{Ba \left( OH \right) _2}_\text{barytová voda} + H_2$ 
    \end{enumerate}
    Sloučeniny Ca (stavebnictví)
    \[\underbrace{CaCO_3}_\text{vápenec} \,\overrightarrow{\, _{800^\circ C} \,} \, \underbrace{CaO}_\text{pálené vápno} + CO_2\]
    \newline
    \[CaO + 2\WATER \,\to\, \underbrace{Ca \left(OH\right)_2}_\text{hašené vápno}\]
    \newline
    \[Ca\left(OH\right)_2 + \underbrace{CO_2 \downarrow}_{ze~vzduchu} ~\to~ CaCO_3 + \WATER\] ...princip tvrdnutí malty
    \newline \newline
    \underline{Podstata krasových jevů:} Uhličitany jsou ve vodě nerozpustné, ale v přítomnosti $CO_2$ (ze vzduchu) se rozpouštějí:
    \[CaCO_3 + CO_2 + \WATER ~\rightleftharpoons ~ Ca \left( HCO_3 \right)_2 \]
    Zpětná rekristalizace na $CaCO_3$ = minerál \underline{sintr} - krápníky
    \begin{description}
        \item[a)] stalagnit - $\bigwedge$
        \item[b)] stalagtit - $\bigvee$
        \item[c)] stalagnát - spojený \tiny{(..nenašel jsem vhodný znak x, btw proč všichni Češi znají krápníky, ale když se jich zeptáš na prvního prezidenta tak budou tupě čumět.)}
    \end{description}

    \subsubsection*{Význam}
    \begin{description}
        \item[Ca, Mg] - biogenní prvky
        \item[Ca] - kosti, zuby
        \item[Mg] \textbf{- součást molekuly chlorofilu}
        \item[Be] - lehký tvrdý kov (o 30\% lehční než Al), slitiny se používají pro výrobu nástrojů i raket, sloučeniny jsou toxické
    \end{description}

    \subsubsection*{Poznámka}
    Minerál beryl [$3BeO \cdot Al_2O_3 \cdot 6SiO_2$]

    - oxidy smaragd(zelený) a akvamarín(modrý)

\subsection{3. Hlavní podskupina - $p^1$ prvky}
\textbf{B, Al, Ga, In, Th}
"Byl Ale Gagarin Indická Tlama", "Běžela Alena Gálií, Indiáni Táhli jí"

\vspace{1em}

$\underbrace{B}_{nekov}, \underbrace{Al, Ga, In, Tl}_{kovy}$

\vspace{1em}

s $\uparrow$Z: $\uparrow$m, $\uparrow$r, kovový charakter, $\downarrow$elektronegativita

\vspace{1em}

Valenční elektrony: 
\( \underbrace{ns^2}_{\boxed{\uparrow\downarrow}} , \underbrace{np^1}_{\boxed{\downarrow\phantom{\uparrow}}\boxed{\phantom{\downarrow\uparrow}}\boxed{\phantom{\downarrow\uparrow}}} \)
$\; \rightarrow$ \underline{hl}.(nejčastější oxidační \# = III)

\subsection{B (bor, borum, borine (en))}
Vázaný ve sloučeninách, nekovový prvek, málo reaktivní, využívá se jako moderátor v jaderných reaktorech (například v Jaderné elektrárně Temelín)

\subsubsection{Minerály}

\underline{borax} = $Na_2B_4O_7 \; \cdot \; 10H_2O$

v analytiké chemii "boraxová perlička" - při $900^\circ$C $\rightarrow$ sklovitá hmota,
která se v přítomnosti různých iontů zabarvuje:
\begin{itemize}
    \item $Co^{2+}$ ... modrá
    \item $Mn^{2+}$ ... fialová
    \item $Cr^{3+}$ ... zelenáf
\end{itemize}

\subsubsection*{Příprava:}
\[B_2O_3 \; + \; 3Mg \; \longrightarrow \; 2B \; + \; 3MgO\]

\[B_2O_3 \; + \; 2Al \; \longrightarrow \; 2B \; + \; Al_2O_3\] \centering aluminotermie

\raggedright
\subsubsection*{Sloučeniny}
Borany = borovodíky (obecný vzorec $B_nH_{2n+2}$)

\[\underbrace{Mg_3B_2}_\text{borid hořečnatý} \; + \; 6H_2O \; \longrightarrow \; 3Mg\left(OH\right)_2 \; + \; \underbrace{B_2H_6}_\text{diboran - plyn}\]
\smallskip

$H_3BO_3$ - kys trihydrogenboritá (ortoboritá), její 3\%$\odot$ je "borová voda"
\smallskip

$H_3BO_3 \; \overrightarrow{\text{\tiny{var, -$H_2O$}}} \; HBO_2$ (kyselina hydrogen boritá)
\smallskip

BN - nitrid boru
\smallskip

$B_4C$ - karbid boru - brusný materiál, velmi tvrdá černá krystalická látka.
Používá se na výrobu neprůstřelných oděvů, brzdová a spojková obložení, nejtvrdší látka na zemi - brousí i diamanty

\subsection{Al (hliník, aluminium)}
\smallskip

3. nejrozšířenější prvek zemské kůry (8.3\%) - první je kyslík, druhý křemík
\smallskip

je složkou vyvřelých minerálů \textbf{živce, slídy}, kaolinit, kryolit ($Na_3\left[AlF_6\right]$), granát ($Ca_3Al_2\left(SiO_3\right)_3$),
beryl ($Be_3Al_2Si_6O_{18}$), tyrkys ($Al_2\left(OH\right)_3PO_4H_2OCu$), korund a jeho obdoby \textbf{rubín, safír, smaragd}
\smallskip

\textbf{Bauxit} (AlO, je to směs oxidů hliníku a trochy železa), těží se v Austrálii, Brazílii, na Jamajce - \textbf{vyrábí se z něj Al}

\subsubsection*{Výroba}
\begin{enumerate}
    \item \textbf{Bayerový způsob}
        - elektrolýza bauxitu při $980^\circ$C, elektrody z uhlíku.
        Na katodě ($K^-$) se vylučuje Al.
        Na anodě ($A^+$) dochází ke spalování uhlíku kyslíkem na CO a $CO_2$
    \item  z přímo z bauxitu
\end{enumerate}

\subsubsection*{Vlastnosti}
poměrně reaktivní ($2Al \; + \; 6HCl \; \longrightarrow \; 2AlCl_3 \; + \; 3H_2$)

stříbrný kov, lehký, přijatelný vodič elektřiny

odolný proti korozi (na povrchu vrstvička $Al_2O_3$), tažný (alobal), snadno se tvoří slitiny

\subsubsection{Použití}
\begin{itemize}
    \item konstrukční kov (letadla, vesmírné lodě)
    \item protikorozivní prvek ($Al_2O_3$) - takzvaná pacivizace kovů
    \item \textbf{aluminotermie} - silné redukční účinky práškovitého Al:
\end{itemize}

\begin{multicols}{2}
    \chemfig{3SiO_2 + 4Al \to 3Si + 2Al_2O_3}

    \chemfig{Co_2O_3 + 2Al \to 2Co + Al_2O_3}
    
    \chemfig{Cr_2O_3 + 2Al \to 2Cr + Al_2O_3}

    \chemfig{3Mn_3O_4 + 8Al \to 9Mn + 4Al_2O_3}

    \chemfig{\underbrace{Fe_2O_3 + 2Al}_{Termit} \to 2Fe + Al_2O_3}
\end{multicols}

Sloučenina \chemfig{Al_2O_3} - bílý prášek (žáruvzdorný, materiál v nehořlavých cihlách), brusný materiál

$3Na_2SO_4 \; + \underbrace{2Al(OH)_3}_\text{amfoterní charakter} \longrightarrow  \; Al_2(SO_4)_3 \; + \; 6NaOH$

\vspace{1em}

$Al(OH)_3 \; + \; 3HNO_3 \; \to \; Al(NO_3)_3 \; + \; 3H_2O$

$Al(OH)_3 \; + \; KOH \; \to \;  K[Al(OH)_4]$